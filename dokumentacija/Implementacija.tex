\chapter{Implementacija i korisničko sučelje}
		
		
		\section{Korištene tehnologije i alati}
		
			%\textbf{\textit{dio 2. revizije}}
			 Komunikacija u timu realizirana je korištenjem aplikacije Discord\footnote{https://discord.com/}, a kao sustav za upravljanje izvornim kodom Git\footnote{https://git-scm.com/}. Udaljeni repozitorij projekta je dostupan na web platformi GitHub\footnote{https://github.com/}.
			 
			 Kao uređivač izvornog koda korišten je Visual Studio Code\footnote{https://code.visualstudio.com/} napravljen od strane Microsofta s Electron Frameworkom, za Windows, Linux i macOS. Dolazi s ugrađenom podrškom za JavaScript, TypeScript i Node.js te ima bogat ekosustav proširenja za druge jezike i runtimeove. Neke od značajki uključuju podršku za debugiranje, isticanje sintakse, inteligentno dovršavanje koda, isječke, refaktoriranje koda i ugrađeni Git.
			 
			 Aplikacija je napisana koristeći web okvir Flask\footnote{https://flask.palletsprojects.com/en/3.0.x/}. Razvijen je od strane Armina Ronachera, vođe Međunarodne grupe entuzijasta za Python. Temelji se na WSGI alatima i Jinja2 predlošku. Ovaj okvir pokriva širok spektar primjene, od osnovnih koncepta kao što su postavljanje i instalacija do naprednijih koncepta poput autentifikacije korisnika i integracije baze podataka.
			 
			 Za izradu frontenda korišten je React\footnote{https://reactjs.org/} i jezik JavaScript\footnote{https://www.javascript.com/}. React, također poznat kao React.js ili ReactJS, je biblioteka u JavaScriptu za izgradnju korisničkih sučelja. Održavana je od strane Facebooka. React se najčešće koristi kao osnova u razvoju web ili mobilnih aplikacija. Složene aplikacije u Reactu obično zahtijevaju korištenje dodatnih biblioteka za interakciju s API-jem.		 
			 
			 Za izradu dokumentacije korišten je LateX\footnote{https://www.latex-project.org/}, markup jezik koji svoju osnovnu primjenu nalazi u izradi znanstvenih publikacija. Osnovna mu je značajka da pisac koristi konvencije označavanja koje predstavljaju ugrađene naredbe za definiranje opće strukture dokumenta, stiliziranje teksta, dodavanje citata, unakrsnih referenci i sl.Tako stvorenu LaTeX datoteku obrađuje softver zvan TeX engine koji koristi ugrađene naredbe kako bi vodio i kontrolirao proces izgradnje profesionalno složenenog PDF dokumenta.
			\eject 
		
	
		\section{Ispitivanje programskog rješenja}
			
			\textbf{\textit{dio 2. revizije}}\\
			
			 \textit{U ovom poglavlju je potrebno opisati provedbu ispitivanja implementiranih funkcionalnosti na razini komponenti i na razini cijelog sustava s prikazom odabranih ispitnih slučajeva. Studenti trebaju ispitati temeljnu funkcionalnost i rubne uvjete.}
	
			
			\subsection{Ispitivanje komponenti}
			\textit{Potrebno je provesti ispitivanje jedinica (engl. unit testing) nad razredima koji implementiraju temeljne funkcionalnosti. Razraditi \textbf{minimalno 6 ispitnih slučajeva} u kojima će se ispitati redovni slučajevi, rubni uvjeti te izazivanje pogreške (engl. exception throwing). Poželjno je stvoriti i ispitni slučaj koji koristi funkcionalnosti koje nisu implementirane. Potrebno je priložiti izvorni kôd svih ispitnih slučajeva te prikaz rezultata izvođenja ispita u razvojnom okruženju (prolaz/pad ispita). }
			
			
			
			\subsection{Ispitivanje sustava}
			
			 \textit{Potrebno je provesti i opisati ispitivanje sustava koristeći radni okvir Selenium\footnote{\url{https://www.seleniumhq.org/}}. Razraditi \textbf{minimalno 4 ispitna slučaja} u kojima će se ispitati redovni slučajevi, rubni uvjeti te poziv funkcionalnosti koja nije implementirana/izaziva pogrešku kako bi se vidjelo na koji način sustav reagira kada nešto nije u potpunosti ostvareno. Ispitni slučaj se treba sastojati od ulaza (npr. korisničko ime i lozinka), očekivanog izlaza ili rezultata, koraka ispitivanja i dobivenog izlaza ili rezultata.\\ }
			 
			 \textit{Izradu ispitnih slučajeva pomoću radnog okvira Selenium moguće je provesti pomoću jednog od sljedeća dva alata:}
			 \begin{itemize}
			 	\item \textit{dodatak za preglednik \textbf{Selenium IDE} - snimanje korisnikovih akcija radi automatskog ponavljanja ispita	}
			 	\item \textit{\textbf{Selenium WebDriver} - podrška za pisanje ispita u jezicima Java, C\#, PHP koristeći posebno programsko sučelje.}
			 \end{itemize}
		 	\textit{Detalji o korištenju alata Selenium bit će prikazani na posebnom predavanju tijekom semestra.}
			
			\eject 
		
		
		\section{Dijagram razmještaja}
			
			\textbf{\textit{dio 2. revizije}}
			
			 \textit{Potrebno je umetnuti \textbf{specifikacijski} dijagram razmještaja i opisati ga. Moguće je umjesto specifikacijskog dijagrama razmještaja umetnuti dijagram razmještaja instanci, pod uvjetom da taj dijagram bolje opisuje neki važniji dio sustava.}
			
			\eject 
		
		\section{Upute za puštanje u pogon}
		
			\textbf{\textit{dio 2. revizije}}\\
		
			 \textit{U ovom poglavlju potrebno je dati upute za puštanje u pogon (engl. deployment) ostvarene aplikacije. Na primjer, za web aplikacije, opisati postupak kojim se od izvornog kôda dolazi do potpuno postavljene baze podataka i poslužitelja koji odgovara na upite korisnika. Za mobilnu aplikaciju, postupak kojim se aplikacija izgradi, te postavi na neku od trgovina. Za stolnu (engl. desktop) aplikaciju, postupak kojim se aplikacija instalira na računalo. Ukoliko mobilne i stolne aplikacije komuniciraju s poslužiteljem i/ili bazom podataka, opisati i postupak njihovog postavljanja. Pri izradi uputa preporučuje se \textbf{naglasiti korake instalacije uporabom natuknica} te koristiti što je više moguće \textbf{slike ekrana} (engl. screenshots) kako bi upute bile jasne i jednostavne za slijediti.}
			
			
			 \textit{Dovršenu aplikaciju potrebno je pokrenuti na javno dostupnom poslužitelju. Studentima se preporuča korištenje neke od sljedećih besplatnih usluga: \href{https://aws.amazon.com/}{Amazon AWS}, \href{https://azure.microsoft.com/en-us/}{Microsoft Azure} ili \href{https://www.heroku.com/}{Heroku}. Mobilne aplikacije trebaju biti objavljene na F-Droid, Google Play ili Amazon App trgovini.}
			
			
			\eject 