\chapter{Zaključak i budući rad}
		
		%\textbf{\textit{dio 2. revizije}}\\
		
		 %\textit{U ovom poglavlju potrebno je napisati osvrt na vrijeme izrade projektnog zadatka, koji su tehnički izazovi prepoznati, jesu li riješeni ili kako bi mogli biti riješeni, koja su znanja stečena pri izradi projekta, koja bi znanja bila posebno potrebna za brže i kvalitetnije ostvarenje projekta i koje bi bile perspektive za nastavak rada u projektnoj grupi.}
		
		 %\textit{Potrebno je točno popisati funkcionalnosti koje nisu implementirane u ostvarenoj aplikaciji.}
Projektni zadatak našeg tima bio je razvoj web aplikacije za promociju događanja. Korisnici ovisno o ulozi posjetitelja, organizatora ili administratora mogu izvoditi različite radnje nad vlastitim i tuđim objavama. Provedba projekta odvijala se u dvije faze. 

Prva faza projekta uključivala je okupljanje tima za razvoj aplikacije, upoznavanje s
projektnim zadatka i intenzivan rad na dokumentiranju zahtjeva koje je naš krajnji projekt morao ispuniti. Kvalitetna provedba prve faze uvelike je olakšala daljnji rad pri realizaciji osmišljenog sustava jer je cjelokupnu problemtiku uspješno podijelila na više manjih problema.
Izrađeni obrasci i dijagrami (obrasci uporabe, sekvencijski dijagrami, model baze
podataka, dijagram razreda) bili su od pomoći u kasnijoj fazi implementacije. 

Druga faza projekta, iako nešto kraća od prve, bila je puno intenzivnija po pitanju samostalnog rada članova. Raznolikost među iskustvima članova tima u izradi sličnih implementacijskih rješenja primorao je iskusnije članove da svoja znanja adekvatno prenesu na kolege koje nisu imali doticaja s takvim izazovima. Poseban naglaskak bio je na samostalnom učenju odabranih alata i
programskih jezika kako bi svi uspješno ispunili zadatke postavljene pred njih. Osim realizacije rješenja,
u drugoj fazi je bilo potrebno dokumentirati ostale UML dijagrame i izraditi popratnu dokumentaciju kako bi budući korisnici mogli lakše koristiti ili vršiti pre-
inake na sustavu. Dobro izrađen kostur projekta uštedio nam je mnogo vremena
prilikom izrade aplikacije te smo izbjegli moguće pogreške u izradi koje bi bile
vremenski skupe za ispravljanje u daljnjoj fazi projekta. Podijela zadataka bazirala se po funkcionalnostima koje je potrebno implementirati tako da je svaki član imao doticaja s cjelokupnom infrastrukturom.

Komunikacija među članovima odvijala se putem Discorda čime smo postigli efikasnu razmjenu znanja i savjeta za realizaciju zadataka postavljenih pred svakog člana. Moguće proširenje postojeće inačice sustava je izrada mobilne aplikacije čime bi cilj projektnog zadatka
bio ostvaren u većoj mjeri nego s web aplikacijom. Od tehničkih izazova s kojima smo se susreli posebno bi izdvojili efikasnu pohranu i dohvaćanje korisnički generiranog sadržaja poput slika. Adekvatnom komunikacijom i istraživačkim radom, rješenje smo pronašli u usluzi Firebase koja je podigla stupanj iskoristivosti naše aplikacije na višu razinu.

Sudjelovanje na ovakvom projektu bilo je vrijedno iskustvo svim članovima
tima jer smo kroz cjelokupni period rada na projektu iskusili važnost zajedničke suradnje te stekli nove socijalne i profesionalne veze koje će nam biti od velike koristi u karijernom razvoju. Također, osjetili smo važnost dobre vremenske organiziranosti i koordiniranosti između članova tima te smo izuzetno zadovoljni postignutim rezultatima.
		\eject 